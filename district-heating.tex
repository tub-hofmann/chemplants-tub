\documentclass[%
%10pt,
%varwidth=false,
%crop=true,
border={0mm 0mm 0mm 0mm}]{standalone}
\usepackage[T1]{fontenc}
\usepackage[utf8]{inputenc}
\usepackage[auto]{microtype}
%\usepackage{cmbright}
\usepackage{arev}
\usepackage{amsmath, amssymb, amsfonts, icomma}
\usepackage[version=4]{mhchem}
\usepackage{tikz}
\usepackage{chemplants-tub}
\usepackage{xcolor}
%define stream tip, default is stealth
\setchpstreamtip{latex}
\setchpmainstreamthickness{thick}
%\setchpunitthickness{very thick}

\pgfdeclarelayer{bg}    % declare background layer
\pgfsetlayers{bg,main}  % set the order of the layers (main is the standard layer)

\definecolor{signalgruen}{RGB}{49,127,67}    % RAL 6032 Wasser
\definecolor{signalrot}{RGB}{155,36,35}      % RAL 3001 Dampf
\definecolor{signalgrau}{RGB}{150,153,146}   % RAL 7004 Luft
\definecolor{signalgelb}{RGB}{229,190,001}   % RAL 1003 brennbare und nicht brennbare Gase
\definecolor{signalorange}{RGB}{208,93,40}   % RAL 2010 Säuren
\definecolor{signalviolett}{RGB}{132,76,130} % RAL 4008 Laugen
\definecolor{signalbraun}{RGB}{121,77,62}    % RAL 8002 brennbare und nicht brennbare Flüssigkeiten
\definecolor{signalblau}{RGB}{30,45,110}     % RAL 5005 Sauerstoff

% TABLEAU-10
\definecolor{Tab10-A}{RGB}{78, 121, 167}
\definecolor{Tab10-B}{RGB}{242, 142, 43}
\definecolor{Tab10-C}{RGB}{225, 87, 89}
\definecolor{Tab10-D}{RGB}{118, 183, 178}
\definecolor{Tab10-E}{RGB}{89, 161, 79}
\definecolor{Tab10-F}{RGB}{237, 201, 72}
\definecolor{Tab10-G}{RGB}{176, 122, 161}
\definecolor{Tab10-H}{RGB}{255, 157, 167}
\definecolor{Tab10-I}{RGB}{156, 117, 95}
\definecolor{Tab10-J}{RGB}{186, 176, 172}

\begin{document}
\begin{tikzpicture}

%help grid and labels
%\draw[help lines] (-2,-1) grid (13,13);
%\foreach \pos in {-2,-1,0,1,2,3,4,5,6,7,8,9,10,11,12,13}
%\draw[shift={(\pos,-1)}] (0pt,2pt) -- (0pt,-2pt) node[below] {$\pos$};
%\foreach \pos in {-1,0,1,2,3,4,5,6,7,8,9,10,11,12,13}
%\draw[shift={(-2,\pos)}] (2pt,0pt) -- (-2pt,0pt) node[left] {$\pos$};

%%% components %%%
%production cite A
\pic[xscale=0.25,yscale=1.25,Tab10-A] (a1) at (2,8.5) {block};
\node [align=center] at (a1-anchor) {\footnotesize $\text{A}_1$};
\pic[xscale=0.25,yscale=1.25,Tab10-B] (a2) at (3,8.5) {block};
\node [align=center] at (a2-anchor) {\footnotesize $\text{A}_2$};
%\pic[xscale=0.25,yscale=1.25,Tab10-C] (a3) at (4,8.5) {block};
\node [align=center] at (4,8.5) {\footnotesize $\ldots$};
\pic[xscale=0.25,yscale=1.25,Tab10-E] (a4) at (5,8.5) {block};
\node [align=center] at (a4-anchor) {\footnotesize $\text{A}_k$};
\draw[dashed] (1,9.75) rectangle (6,7);
\node [below] at (3.5,7.5) {\footnotesize Production site A};

%production cite B
\pic[xscale=0.25,yscale=0.5,Tab10-A] (b1) at (3,6) {block};
\node [align=center] at (b1-anchor) {\footnotesize $\text{B}_1$};
\pic[xscale=0.25,yscale=0.5,Tab10-B] (b2) at (4,6) {block};
\node [align=center] at (b2-anchor) {\footnotesize $\text{B}_2$};
%\pic[xscale=0.25,yscale=0.5,Tab10-C] (b3) at (5,6) {block};
\node [align=center] at (5,6) {\footnotesize $\ldots$};
\pic[xscale=0.25,yscale=0.5,Tab10-E] (b4) at (6,6) {block};
\node [align=center] at (b4-anchor) {\footnotesize $\text{B}_l$};
\draw[dashed] (2,6.5) rectangle (7,5);
\node [below] at (4.5,5.5) {\footnotesize Production site B};

%production cite C
\pic[xscale=0.25,yscale=0.5,Tab10-A] (c1) at (4,4) {block};
\node [align=center] at (c1-anchor) {\footnotesize $\text{C}_1$};
\pic[xscale=0.25,yscale=0.5,Tab10-B] (c2) at (5,4) {block};
\node [align=center] at (c2-anchor) {\footnotesize $\text{C}_2$};
%\pic[xscale=0.25,yscale=0.5,Tab10-C] (c3) at (6,4) {block};
\node [align=center] at (6,4) {\footnotesize $\ldots$};
\pic[xscale=0.25,yscale=0.5,Tab10-E] (c4) at (7,4) {block};
\node [align=center] at (c4-anchor) {\footnotesize $\text{C}_m$};
\draw[dashed] (3,4.5) rectangle (8,3);
\node [below] at (5.5,3.5) {\footnotesize Production site C};

%heat storage
\pic[xscale=0.33,yscale=0.33] (hs) at (8,8) {dome tank};
\node [below] at (hs-bottom) {\footnotesize Heat storage};

%merge
\pic[rotate=180] (m1) at (8,9.5) {valve triple=main};

%heat demand
\node [left, anchor=west] at (10,9.5) {\footnotesize Heat demand A};
\node [left, anchor=west] at (10,6) {\footnotesize Heat demand B};
\node [left, anchor=west] at (10,4) {\footnotesize Heat demand C};

%heat streams
\draw [main stream, signalrot] (6,9.5) -- (m1-right);
\draw [main stream, signalrot] (m1-top) -- (hs-top);
\draw [main stream, signalrot] (hs-top) -- (m1-top);
\draw [main stream, signalrot] (m1-left) -- (10,9.5);

\draw [main stream, signalrot] (7,6) -- (10,6);
\draw [main stream, signalrot] (8,4) -- (10,4);

%electricity demand
\node [left, anchor=west] at (10,2.25) {\footnotesize Electricity};


%electricity streams
\draw [main stream, signalgrau] (1.5,7) |- (10,2);
\draw [main stream, signalgrau] (2.5,5) |- (10,2.25);
\draw [main stream, signalgrau] (3.5,3) |- (10,2.5);

%overall system
\draw[dashdotted] (0.75,10) rectangle (9.25,1.75);

%input streams
\draw [main stream, signalgelb] (-1.5,7) -- (0.75,7);
\node [right, anchor=south west] at (-1.5,7) {\footnotesize Natural Gas};

\draw [main stream, signalgrau] (-1.5,6) -- (0.75,6);
\node [right, anchor=south west] at (-1.5,6) {\footnotesize Electricity};

\draw [main stream, signalgruen] (-1.5,5) -- (0.75,5);
\node [right, anchor=south west] at (-1.5,5) {\footnotesize Biomass};

\end{tikzpicture}

\end{document}