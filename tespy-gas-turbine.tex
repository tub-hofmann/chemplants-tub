\documentclass[%
%10pt,
%varwidth=false,
%crop=true,
border={0mm 0mm 0mm 0mm}]{standalone}
\usepackage[T1]{fontenc}
\usepackage[utf8]{inputenc}
\usepackage[auto]{microtype}
%\usepackage{cmbright}
\usepackage{arev}
\usepackage{amsmath, amssymb, amsfonts, icomma}
\usepackage[version=4]{mhchem}
\usepackage{tikz}
\usepackage{chemplants-tub}
\usepackage{xcolor}
%define stream tip, default is stealth
\setchpstreamtip{latex}
\setchpmainstreamthickness{thick}
%\setchpunitthickness{very thick}

\pgfdeclarelayer{bg}    % declare background layer
\pgfsetlayers{bg,main}  % set the order of the layers (main is the standard layer)

\definecolor{signalgruen}{RGB}{49,127,67}    % RAL 6032 Wasser
\definecolor{signalrot}{RGB}{155,36,35}      % RAL 3001 Dampf
\definecolor{signalgrau}{RGB}{150,153,146}   % RAL 7004 Luft
\definecolor{signalgelb}{RGB}{229,190,001}   % RAL 1003 brennbare und nicht brennbare Gase
\definecolor{signalorange}{RGB}{208,93,40}   % RAL 2010 Säuren
\definecolor{signalviolett}{RGB}{132,76,130} % RAL 4008 Laugen
\definecolor{signalbraun}{RGB}{121,77,62}    % RAL 8002 brennbare und nicht brennbare Flüssigkeiten
\definecolor{signalblau}{RGB}{30,45,110}     % RAL 5005 Sauerstoff

% TABLEAU-10
\definecolor{Tab10-A}{RGB}{78, 121, 167}
\definecolor{Tab10-B}{RGB}{242, 142, 43}
\definecolor{Tab10-C}{RGB}{225, 87, 89}
\definecolor{Tab10-D}{RGB}{118, 183, 178}
\definecolor{Tab10-E}{RGB}{89, 161, 79}
\definecolor{Tab10-F}{RGB}{237, 201, 72}
\definecolor{Tab10-G}{RGB}{176, 122, 161}
\definecolor{Tab10-H}{RGB}{255, 157, 167}
\definecolor{Tab10-I}{RGB}{156, 117, 95}
\definecolor{Tab10-J}{RGB}{186, 176, 172}

\begin{document}
\begin{tikzpicture}

%help grid and labels
%\draw[help lines] (-2,-1) grid (8,5);
%\foreach \pos in {-2,-1,0,1,2,3,4,5,6,7,8}
%\draw[shift={(\pos,-1)}] (0pt,2pt) -- (0pt,-2pt) node[below] {$\pos$};
%\foreach \pos in {-1,0,1,2,3,4,5}
%\draw[shift={(-2,\pos)}] (2pt,0pt) -- (-2pt,0pt) node[left] {$\pos$};

%components
\pic[scale=0.75,rotate=90] (airsrc) at (1,-1) {inlet};
\node [left] at (airsrc-top) {\footnotesize SRC-A};
\pic[scale=0.5] (co) at (1,1) {compressor-iso};
\node [left] at (co-left) {\footnotesize AC};
\pic[scale=0.75,rotate=270] (fuelsrc) at (3,4.5) {inlet};
\node [left] at (fuelsrc-bottom) {\footnotesize SRC-F};
\pic (cc) at (3,2.5) {combustion-chamber-din};
\node [below] at (cc-ash outlet) {\footnotesize CC};
\pic[xscale=1.25,yscale=2] (exp) at (5.5,1) {turbine};
\node [right] at (exp-outlet top) {\footnotesize EXP};
\pic[scale=0.5] (gen) at (7.5,1) {generator-iso};
\pic[scale=0.75,rotate=270] (exhaustsnk) at (6,-1) {outlet};
\node [right] at (exhaustsnk-top) {\footnotesize SNK-E};

%streams
\draw [main stream] (airsrc-stream) -- (co-inlet);
\draw [main stream] (co-outlet) |- (cc-oxidant inlet);
\draw [main stream] (fuelsrc-stream) -- (cc-fuel inlet);
\draw [main stream] (cc-exhaust outlet) -| (exp-inlet top);
\draw [main stream] (exp-outlet bottom) -- (exhaustsnk-stream);

\draw [solid] (co-right) -- (exp-left);
\draw [solid] (exp-right) -- (gen-left);

\draw [utility stream] (gen-right) -- (9,1);
\node [right] at (9,1) {\footnotesize $\dot W_\text{el}$};

%nodes
\node [left] at (1,-0.5) {\footnotesize 1};
\node [left] at (1,2.5) {\footnotesize 2};
\node [left] at (3,4) {\footnotesize 3};
\node [right] at (5,2.5) {\footnotesize 4};
\node [right] at (6,-0.5) {\footnotesize 5};

%%% system boundary %%%
\draw[dashed] (-0.25,3.5) rectangle (8.5,-0.25);
\node [above left, anchor=south east] at (8.5,3.5) {\footnotesize System Boundary};

%%% legend %%%%
\node [align=left,anchor=north west] at (10.5,3) {\footnotesize 
	\begin{tabular}{ll}
    AC & Air Compressor \\
    CC & Combustion Chamber \\
    EXP & Expander \\
    G & Generator \\
    SRC-A & Source, Air \\
    SRC-F & Source, Fuel \\
    SNK-E & Sink, Exhaust \\
    \end{tabular}
};	



\end{tikzpicture}

\end{document}